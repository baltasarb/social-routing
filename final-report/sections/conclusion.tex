\chapter{Conclusion}
All the functionalities are currently done at this point. On the client side all the application flow is done properly, containing the authentication
that communicates with the Social Routing Service and Google Sign In API, the route search used to find the best possible route by getting 
authentication information from our service, and the route creation/update that is fully done in the client side and sent to the SRS to be 
saved and finally the user profile to show the user created routes.The Server side is also fully developed to materialize all the 
functionalities established initially like the support of the model and its management, server pagination and error handling. 
The focus on this system is to expose a good route search, to find the route that is the most relevant to the user and the Live Tracking that 
indicates the path to the route and understands if the user is inside the route while he is traveling in it.

While implementing the Client Application some problems were encountered when utilizing the Google Maps API, namely the API key and the 
service's usage. Without an account with a credit card associated we were limited to one API call per day, which was limiting our testing 
capabilities. This was overcome by creating a wallet (offered by the service with an initial value). The process that is required to create 
a route was also a problem, there are some considerations to be had due to the nature of a path: where it starts, where it ends, if it is 
circular, if it is doable in a determined time amongst others. This topic is intertwined with route definition, changes in the definition 
also change the creation. The Service had problems initially with the connection to the Database Management System, specifically because 
the choice of technologies was not the appropriate one, which forced a shift to a different one. This used valuable time because two different 
libraries were learned (Spring Data and JDBI) and later Spring Data was discarded. All the system is doing the purpose, with some optimization 
in some functionalities like the search and the live tracking. 