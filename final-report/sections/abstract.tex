\begin{abstract}
    \begin{center}\LARGE\bfseries
        Abstract
        \end{center}
This project consists of a mobile application to allow an application user to 
either create or undergo pedestrian routes. A route is a set of geographical coordinates
(latitude and longitude) and as such a route is a path formed by sub paths on a map.

The application is structured in two major components, one with the front end, the Social Routing Client
Application (SRCA), built for Android based devices and the back end, the Social Routing 
Service (SRS), which exposes it's functionality through an HTTP\cite{httponlinedocs} API\cite{api}.

The SRCA is made for Android users, using Kotlin\cite{kotlinwebsite} as a programming language,
and the guidelines made available by Google for Android application development.
The features available are user login, profile viewing, route creation, searching, and live tracking.
The live tracking feature helps the user while he is performing a route, informing the user
about his current location and if he is far from the route. There is also the ability
to rate other users route's.

The SRS exposes it's functionality through an HTTP API, and is subdivided into a Spring Application\cite{springwebsite}, using
Kotlin as a programming language, and a Database Management System, using PostgreSQL\cite{postgresql}. The database is necessary
because as a user creates routes, other users must be able to see them and as such storage is required for
persistent information maintainability.

As the application requires users to access information such as routes created by
other users as well as their own, it requires authentication. Both the SRCA and the SRS require the user
to be authenticated, being that they communicate with each other, the user requests credentials from the 
SRCA that in turn reaches out to the SRS to get them. This is made with 
aid of an external API being the SRS responsible for the generation of user credentials.

The major problems encountered were the route search functionality: a user has
the ability to search for routes near his location, and authentication. The first 
was solved with aid from a PostgreSQL extension PostGIS\cite{postgis} and the second with the Google
Sign In API\cite{googlesignindocs}.
\end{abstract}
