\chapter{Authentication}

The nature of the project requires that user information such as user made routes is stored and
because of this user authentication is required. 

To enable authentication the Google Sign In API was the chosen technology. Mainly because the client side 
application is being built for the Android Operating System, and any user who has an Android based device 
will already have his credentials stored in the device, guaranteeing a very easy way both to register and 
login into the Social Routing Application.

From a technical point of view, user authentication is handled in two major components, the Social Routing 
Client Application and the Social Routing Service. 

\section{Social Routing Client Application}
    The client application is responsible for two stages of authentication, the user registration and the continuous
    validation of its requests to the Social Routing Service.
    \subsection{Registration}
        When a user first logs in into the Social Routing Client Application, it is required that he inputs his google
        login credentials in an activity provided by google that is used to help with the Google Sign In API communication.
        After the credentials are inserted a request is made for the user google account's corresponding id token.
        This token is then sent to the Social Routing Service that will in turn register the new user and return an
        access token used to authenticate that user from then on with the service. 
    \subsection{Request Validation}
        After being registered the user in question will have an access token that allows requests to be made to 
        the Social Routing Service. This token is sent in each request made to the Social Routing Service, which validates
        it and allows the request to go through if it is valid.  

\section{Social Routing Service}
    Like the client application the Social Routing Service contains two stages of authentication, but in the request validation
    it also handles a particular case, access token refreshing. 
    \subsection{Registration}
    The first step is the validation of the google id token, made with an API call. This call returns an object containing
    relevant user information such as the user name or email. The only information stored of the id token is however the field
    subject which is a unique user identifier on the google platform.   

    When this subject is retrieved, confirming the token's validity, it is then paired with both an access token and refresh 
    token, and together they form the user authentication information. Both the access token and refresh token are generated
    by the Social Routing Service, using the Secure Random java class. Upon generation both of them are hashed and stored 
    in the database along with the subject, being that on a registration request, they are also return in the body of an HTTP
    response, in their original form, to the registration request maker. 
    
    \subsection{Request Validation}
    When any request is made to the Social Routing Service the access token must be added to the headers of the HTTP request,
    specifically, the Authorization header. This information is retrieved by the server in each request and checked against 
    the database. If present then the request is valid, if not, forbidden. Both the refresh and access token have the expiration
    time of one day. Because of this, when a request is made with an expired access token, a response that is is invalid
    is sent and the request maker must then ask for an access token renewal with his refresh token. 
